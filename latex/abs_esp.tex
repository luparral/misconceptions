%\begin{center}
%\large \bf \runtitulo
%\end{center}
%\vspace{1cm}
\chapter*{\runtitulo}

\noindent Las \textit{misconceptions} son ideas o razonamientos que, si bien poseen una determinada lógica y coherencia que los hace verosímiles, proveen a la persona que las posee un entendimiento incorrecto de un determinado fenómeno o evento. Esta verosimilitud es lo que hace que sean muy difíciles de desarraigar y provoquen problemas en el aprendizaje. Muchos autores han estudiado las \textit{misconceptions} en distintos campos de la ciencia debido a que conocerlas permite elaborar estrategias educativas más eficaces. Sin embargo, el estudio de las \textit{misconceptions} en el área de las Ciencias de la Computación es aún bastante reciente, y está enfocado principalmente a las \textit{misconceptions} en programación y, en menor medida, sobre Internet y sus servicios. En esta tesis investigamos la presencia o no de \textit{misconceptions} en alumnos y alumnas de alrededor de 10 años sobre distintos temas de las Ciencias de la Computación, tales como el almacenamiento de grandes volúmenes de datos en YouTube, la manera en la que se envían los mensajes en WhatsApp y los resultados de compartir archivos en esta plataforma, y la gratuidad de algunas aplicaciones en Internet. Dejamos de lado temas como programación ya que se encuentran más explorados, y priorizamos otros temas que por su cotidianeidad son más relevantes para el grupo estudiado. Realizamos una encuesta que los alumnos completaron dentro del marco de la clase online en presencia del docente y analizamos los datos obtenidos de manera cuantitativa y estadística. 
Observamos que a pesar de que los niños y niñas entrevistados son ávidos consumidores de estas tecnologías poseen \textit{misconceptions} en estos temas.

\bigskip

\noindent\textbf{Palabras claves:} \textit{Misconceptions}, Didácticas Ciencias de la Computación, Nivel primario.