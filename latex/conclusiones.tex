En esta tesis investigamos si niños y niñas de escuelas Argentinas de alrededor de 10 años de edad presentan \textit{misconceptions} en algunos temas de Ciencias de la Computación. En particular, nos centramos en analizar cuáles son sus concepciones respecto a temáticas que les fueran familiares en tanto se encuentran muy presentes en su cotidianidad: dónde se almacenan los videos en YouTube, de qué manera se comparten archivos y cómo viajan los mensajes por WhatsApp y por qué algunas de las aplicaciones que usan diariamente (como TikTok o Instagram, entre otras) son gratis.

Para esto, realizamos una encuesta en la que les planteamos distintas preguntas que les permitieran reflexionar acerca de estos temas. Los chicos y chicas completaron la actividad en un marco de clase online, de manera individual pero con el docente presente. De esta forma, nos aseguramos que sus respuestas fueran lo más “directas” y sin intervención posible, ya que al ser un trabajo en tiempo real, no había posibilidad de hacer preguntas a un adulto o compañero, ni de buscar información en Internet. Una vez que tuvimos sus respuestas, analizamos los datos resultantes para poder entender si había \textit{misconceptions} y cuáles eran estas.

En primer lugar, cabe destacar que gracias a la información poblacional recolectada, pudimos saber que el grupo encuestado estaba ampliamente familiarizado con el conjunto de tecnologías sobre las cuales les preguntamos: son usuarios frecuentes de computadoras y celulares (la mayoría posee computadora en su casa y tiene un celular propio), y utilizan estos dispositivos para variadas actividades, tanto escolares, como recreativas y de socialización. 

A pesar de utilizar YouTube habitualmente, pudimos descubrir que hay \textit{misconceptions} respecto a de qué manera se almacenan los videos en la plataforma: la mayoría de los encuestados piensa que los archivos están guardados en “la nube”. Si bien hubo otras respuestas, pudimos ver que la cantidad de alumnos con alguna \textit{misconception} en este tema fue casi el doble que la de alumnos que respondieron correctamente.

Respecto de la forma en la que se comparten archivos por WhatsApp, observamos que por lo general tienen claro quién tiene acceso a las fotos almacenadas en el propio celular y que al compartir un archivo se envía una copia del mismo. Sin embargo, en el último punto emergió una \textit{misconception}. Cuando les preguntamos si se podía “quitar” el acceso a ese archivo compartido, la mayoría de los encuestados respondió que “bastaba borrar la foto del chat de WhatsApp” para eliminarlo. De esta forma, se hace evidente que la idea de que se ha generado realmente una nueva copia y se ha perdido la propiedad sobre esta no está del todo clara. Este es un punto interesante a trabajar ya que está directamente relacionado con la privacidad de la información compartida y los riesgos que trae la pérdida de la propiedad sobre la misma.

También pudimos ver en nuestros resultados que aún hace falta reforzar los contenidos respecto al funcionamiento de la red de telefonía móvil, ya que la mayoría de los encuestados respondió no saber al respecto, aún siendo que la gran mayoría cuenta con un teléfono celular propio. Un punto llamativo es que, dentro de las opciones que habíamos planteado a la pregunta sobre cómo viajan los mensajes de WhatsApp, habíamos propuesto como una de las opciones “a través de la nube” y esta fue la menos elegida. Esto nos da a entender cuando lo analizamos junto con los resultados de la temática de YouTube comentados anteriormente, que sus concepciones acerca de ``la nube'' tendrían más que ver con un espacio de almacenamiento ``etéreo'' de datos que con un medio por el cual se envía y recibe información. Este punto nos parece interesante para continuar indagando en una siguiente investigación ya que se trata de un término muy presente pero que pareciera tomar distintas interpretaciones, algunas de las cuales podrían tener \textit{misconceptions}.

Sobre la temática referente a la gratuidad de las aplicaciones en Internet pudimos observar que si bien pareciera que muchos de los encuestados tienen concepciones correctas respecto a la pregunta que les planteamos, un gran porcentaje aún conserva una interpretación \textit{naif} o “infantil”. Vale la pena preguntarse aquí si estas \textit{misconceptions} desaparecen y mutan a una visión más adulta sobre cómo funciona el mundo o bien, por el contrario, incluso en los grupos de adultos esta confusión sigue prevaleciendo y falta educar sobre el modelo comercial de las herramientas informáticas de uso cotidiano.

Vale la pena aclarar que las conclusiones mencionadas fueron contrastadas realizando un análisis estadístico que las sustenta.

A futuro y como continuación de este trabajo, sería pertinente ampliar la cantidad de niñas y niños encuestados para poder tener una mejor idea de si los resultados obtenidos son consistentes.

Por otro lado, como mencionamos anteriormente, los alumnos y alumnas encuestados pertenecían a una misma escuela. Consideramos que sería provechoso conseguir la participación de niños y niñas de distintas escuelas de distintos lugares del país para poder analizar los resultados agregando también variables sociales y del contexto educativo de los distintos grupos.

Por último, se podría además variar las edades de los participantes para así entender si hay \textit{misconceptions} que desaparecen o cambian al variar este factor.
