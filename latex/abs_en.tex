%\begin{center}
%\large \bf \runtitle
%\end{center}
%\vspace{1cm}
\chapter*{\runtitle}

\noindent  \textit{Misconceptions} are ideas or reasoning that, although they possess a certain logic and coherence that makes them plausible, provide the person who possesses them with an incorrect understanding of a certain phenomenon or event. This verisimilitude is what makes them very difficult to uproot and causes problems in learning. Many authors have studied \textit{misconceptions} in different fields of science because knowing them allows the development of more effective educational strategies. However, the study of \textit{misconceptions} in the area of Computer Science is still quite recent, and is mainly focused on \textit{misconceptions} in programming and, to a lesser extent, on the Internet and its services. In this thesis we investigate the presence or not of \textit{misconceptions} in students around 10 years old about different topics in Computer Science, such as the storage of large volumes of data on YouTube, the way messages are sent on WhatsApp and the results of sharing files on this platform, and the free nature of some applications on the Internet. We left aside topics such as programming since they are more explored, and prioritized others that due to their everyday relevance are more relevant to the studied group. We conducted a survey that students completed within the online class in the presence of the teacher and we analyzed the data obtained quantitatively and statistically. 
We observed that despite the fact that the children interviewed are avid consumers of these technologies, they have \textit{misconceptions} about these topics.

\bigskip

\noindent\textbf{Keywords:} Misconceptions, Computer Science Didactics, Primary level.