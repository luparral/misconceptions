En este capítulo revisaremos una selección bibliográfica sobre trabajos que se han centrado en la investigación de las concepciones previas de los alumnos.

Distintos autores y autoras han tratado este tema desde diversos enfoques. Los artículos elegidos por nosotros están orientados a darnos un panorama de qué son, de dónde provienen y por qué nos es útil conocerlas. Por otro lado, también nos interesó investigar sobre el estado del arte en el estudio de las concepciones de los jóvenes relativas a Internet, sus características y algunos de sus servicios y funcionalidades, ya que nuestro trabajo se ubicará dentro de este marco teórico.

En el trabajo de Johsua, Samuel y Dupin Jean-Jacques, \textit{Les conceptions des élèves} \cite{johsua}, se explica que las concepciones de los alumnos son razonamientos que muchas veces aparecen relativamente organizados, dotados de una lógica propia, y aptos para aumentar en coherencia interna, todo ello permaneciendo alejado de los modelos científicos. Originándose en la experiencia pasada del sujeto, logran generar una gran resistencia a la enseñanza debido a su plasticidad: al ser eficaces en cierta medida, y capaces de tomar nuevos elementos aportados por la enseñanza para seguir teniendo sentido, se vuelven de alguna manera pertinentes, y es justamente eso lo que los hace tan sólidos.

Respecto a la enseñanza sobre Internet, en su artículo \textit{Students’ mental models of the internet and their didactical exploitation in informatics education}, Papastergiou \cite{papastergiou} dice que es necesario tener en cuenta que los estudiantes no llegan a la escuela como “hojas en blanco”, sino que poseen modelos mentales que se modelan en función de la experiencia personal. Estos modelos constituyen sistemas explicativos que guían la construcción del conocimiento y sus interacciones (con Internet, en el caso del artículo citado).		

En el siguiente cuadro se detallan algunos trabajos relacionados con el estudio de las concepciones de niños y jóvenes respecto de Internet, sus servicios y sistemas computacionales varios.


\begin{table}[H]
\centering
\resizebox{\textwidth}{!}{%
\begin{tabular}{|l|l|l|l|l|}
\hline
Año  & Autor           & Objeto de estudio                               & Edades & Ref \\ \hhline{=====}
2005 & Yan             & Concepciones de Internet                        & 5-12   & \cite{yan}          \\ \hline
2020 & Mertala         & Percepciones de ubiquitous computing e IOT      & 3-6    & \cite{mertala}      \\ \hline
2018 & Edwards et al.  & Conceptos de internet                           & 4-5    & \cite{edwards}      \\ \hline
2017 & Kodama et al.   & Modelos mentales de Google                      & 10-14  & \cite{kodama}       \\ \hline
2005 & Papastergiou    & Modelos mentales de Internet                    & 12-16  & \cite{papastergiou} \\ \hline
2012 & Diethelm et al. & Concepciones de Internet                        & 13-14  & \cite{diethelm}     \\ \hline
2008 & Kafai           & Entendimiento de virus de computadora           & 13-14  & \cite{kafai}        \\ \hline
2019 & Mertala         & Concepciones de Computadoras, Internet y código & 13-14  & \cite{mertala-2}    \\ \hline
2020 & Diethelm et al. & Modelos mentales de sistemas de computadoras    & 10-20  & \cite{pancratz}    \\ \hline

\end{tabular}%
}
\end{table}

El trabajo de Papastergiou \cite{papastergiou} mencionado anteriormente es uno de los más reconocidos en el campo de la educación en las Ciencias de la Computación. En este, la autora estudia a través de un cuestionario de preguntas y dibujos las concepciones de los estudiantes de once escuelas secundarias en Grecia acerca de distintos tópicos de Internet. Su análisis se centra en preguntas como qué es Internet como entidad, cómo está distribuida allí la información, qué estructura tiene, y  en qué se diferencian Internet de la Web. Una de las principales conclusiones de su trabajo es que los estudiantes de secundaria forman estructuras mentales simplistas y utilitarias: no tienen concepciones claras de qué es Internet en tanto entidad física o de los procesos que subyacen a su uso. Los modelos mentales de los estudiantes quedan atados a la perspectiva del usuario, a lo que este ve cuando usa Internet (las interfaces de las aplicaciones de Internet, el contenido de la web y el hardware necesario para conectarse a Internet).

En el trabajo de Diethelm, Wilken y Zumbrägel \cite{diethelm} también se discute la importancia de las concepciones de los alumnos respecto de la Internet a la hora de planear las lecciones de Ciencias de la Computación en las escuelas. Según un modelo llamado ``\textit{Educational Reconstruction}'' \cite{komorek} para el diseño de clases, queda establecido que las perspectivas de los estudiantes son tan importantes como las enunciaciones científicas, puesto que en el proceso en el cual se realiza el aprendizaje debe haber un ida y vuelta entre ambos. En este estudio se intenta responder a las preguntas de cómo alumnos de entre 13 y 14 años explican el funcionamiento de algunos servicios de Internet (en particular e-mail, chat y servicios de streaming) y cómo utilizan los modelos mentales de su vida diaria para este fin. Tras realizar entrevistas semi-estructuradas, se observó que los chicos utilizan distintos modelos para explicar los distintos conceptos y servicios sobre los que fueron preguntados. Estos modelos a su vez pueden ser aprovechados a la hora de enseñar sobre estos temas de manera de enriquecer la adquisición de conocimientos.	
				
Con un interés en los filtros para protección en el uso de Internet, Yan \cite{yan} se ocupó en su trabajo de describir las diferencias que hay según la edad, en el entendimiento de Internet en tanto un sistema híbrido, complejo tanto técnica como socialmente. En su estudio, recolectó información en forma de entrevistas y cuestionarios de un grupo de 111 participantes, agrupados en distintos rangos de edad. Entre sus resultados muestra que los alumnos comienzan a entender Internet como un artefacto complejo cognitiva y socialmente en el rango de entre los 9 y 12 años. A partir de esto, sugiere que los filtros de Internet deberían tener en cuenta a tres grupos: los niños menores a ese rango de edad, que por ser los más vulnerables necesitan filtros más restrictivos. Los de ese rango, al que califica “de transición”, para los cuales no alcanza solamente con filtros, sino a los que también hay que educar para que puedan hacer frente a la experiencia online “indirecta” (a través de sus familias, amigos, etc) y utilizar los servicios de Internet de manera eficaz y segura. Y por último, los usuarios mayores, para los cuales los programas de filtrado no deberían ser utilizados (ya que son lo suficientemente hábiles como para saltearlos), sino que habría que hacer especial hincapié en la educación, en conjunción con monitoreo online para detectar cualquier anomalía especial. 

Edwards et al. \cite{edwards} también se ocuparon de analizar las concepciones acerca de Internet, pero a diferencia de los trabajos mencionados anteriormente, enfocaron su estudio en alumnos más pequeños. Siendo que los niños acceden a dispositivos mobile y touch desde edades cada vez más tempranas, se vuelve de gran importancia que puedan acceder a educación en seguridad online adecuada, y para esto es necesario conocer su entendimiento acerca de Internet. Este paper se basa en las ideas de Vygotsky \cite{vygotsky} acerca de los conceptos cotidianos en los niños pequeños, que derivan de su día a día y las herramientas que utilizan. Estos son contrarios a los conceptos científicos, que proveen información acerca del por qué y cómo las cosas funcionan. Pero de la conjunción de ambos tipos surgen los ``\textit{mature concepts}'', o bien, aquellos conceptos que han atravesado un proceso de maduración. Estos son los que le permiten a un niño razonar desde una perspectiva científica sobre un hecho cotidiano. Bajo esta premisa, los autores intentan descubrir los conceptos cotidianos de un conjunto de 70 niños pequeños, utilizando distintos tipos de herramientas, tales como entrevistas, preguntas sobre una historia o imágenes laminadas y  dibujos. Según ellos, el beneficio de que los conceptos cotidianos de los niños “maduren” es que esto puede hacer que entiendan realmente las implicancias sociales y tecnológicas del uso de Internet.

Otra autora que tomó como grupo de estudio a niños pequeños es Mertala \cite{mertala} en su trabajo ``\textit{Young children's perceptions of ubiquitous computing and the Internet of Things}''. Sin embargo, este trabajo se enfoca en computación ubicua e \textit{Internet of things} y en el entendimiento de los niños acerca de qué objetos cotidianos y tangibles pueden de hecho ser computadoras y estar conectados a Internet. Mertala parte de la premisa teórica de que el aprendizaje ocurre en la interacción con el ambiente social y cultural en el que el niño se encuentra. El conocimiento de los chicos sobre las tecnologías se basa en el contacto que tuvieron con éstas a través de sus padres, hermanos u otras figuras del entorno familiar o escolar. Así, las preguntas que busca responder son cuáles son las percepciones iniciales de los niños pequeños sobre la computación ubicua e \textit{Internet of things} y cómo estas concepciones cambian (o no) luego de que se los introduzca a conceptos científicos sobre estos temas.

En otro trabajo de 2019, la misma autora estudia las concepciones sobre computadoras, código e Internet en niños de entre 5 y 7 años, cómo se relacionan entre sí y de dónde surgen. A diferencia de otros trabajos, en este artículo \cite{mertala-2} se examinan los tres conceptos simultáneamente ya que la autora plantea que aparecen muy profundamente ligados y cualquier división que se haga termina siendo artificial. Entre los resultados obtenidos tras un proceso de DTC (método de \textit{Draw and Tell Conversation} \cite{driessnack}), se puede mencionar que los niños tienen dificultades para distinguir cuándo se encuentran o no online. Esto puede deberse a que la experiencia online actualmente es muy fluida, ya que cuando esta fluidez se perturba (por ejemplo, cuando se experimentan problemas de conexión) es cuando surgen en los chicos cuestionamientos sobre el funcionamiento de Internet. Mertala pudo notar que no había concepciones claras respecto a cómo se relacionan el código con las computadoras, habiendo pocos niños que demostraran tener conocimientos sobre qué es la programación. Además, y a pesar de que el grupo estudiado es considerado el de los “nativos digitales”, observó que sus conocimientos son adquiridos mayormente en el hogar, en función de la observación a sus padres y cuidadores.

Por su parte, Kodama et al. \cite{kodama} enfocan su trabajo en elucidar los modelos mentales sobre el funcionamiento de Google en alumnos de entre 10 y 14 años. A diferencia de los otros trabajos mencionados anteriormente, estos autores encaran su investigación con una pregunta mucho más específica: ¿cómo piensan los estudiantes que funciona Google “detrás de escena”? Es interesante la definición de modelos mentales que toman de  Norman \cite{norman}, según la cual un modelo mental es una representación cognitiva de una persona sobre un sistema que incorpora sus creencias, sin tener en cuenta necesariamente la manera en que el sistema en sí funciona. Entender los modelos mentales de una persona sobre un sistema puede servir para explicar y predecir las maneras en la que él o ella va a interactuar con dicho sistema. Siguiendo esta idea, es que como resultado de este trabajo se indica que el esfuerzo en revisar cómo se explica el funcionamiento de los motores de búsqueda a los estudiantes debe ser tanto a nivel escolar como a nivel de los diseñadores y desarrolladores de las interfaces de usuario de Google, haciendo que estas sean más transparentes y dignas de confianza a los ojos del usuario.

En Kafai \cite{kafai} se examina el entendimiento sobre los virus de computadora,  a través de una epidemia virtual en el juego online Whyville.net. En este estudio se intenta establecer un paralelismo entre esta epidemia virtual y su transmisión a través de un virus, y los virus de computadora, para entender qué conocimientos tienen y qué concepciones manejan sobre este tema los usuarios de este juego (285 jugadores de entre 6 y 18 años). Como resultado se observó que la mayoría de los encuestados poseía un entendimiento muy vago sobre el tema, con concepciones antropomórficas o \textit{naïve} enfocadas en el comportamiento de los virus (es decir, qué tipo de problemas causan en una computadora). 

Por último, cabe mencionar el estudio de Diethelm y Pancratz \cite{pancratz} que se destaca por su enfoque orientado al ``\textit{part-whole-thinking}''. Lo que se quiere investigar es la habilidad de los alumnos de pensar en las partes y en el todo, es decir el proceso de identificación de las partes individuales que conforman un sistema y su interrelación. Según la literatura del campo de la ciencia cognitiva, a través de este procedimiento aprendemos cómo los objetos, sistemas y procesos funcionan. Este conocimiento es la base para comprender nuevos objetos y sistemas, y esta habilidad es de fundamental importancia para entender sistemas computacionales. En este estudio, los autores solicitan a 68 estudiantes de secundaria que dibujen sus concepciones sobre cómo tres sistemas computacionales son por dentro (\textit{smartphones}, consolas de videojuegos y aspiradoras robot). Las habilidades cognitivas que busca encontrar son: la capacidad de identificar partes individuales en un sistema, el entendimiento de sus funcionalidades independientes dentro de ese sistema, cuál es el rol de cada parte del sistema y la relación de las partes entre sí y la capacidad de extrapolar a otros sistemas.
