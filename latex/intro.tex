Las \textit{misconceptions} o concepciones erróneas pueden ser descritas como ideas que proveen un entendimiento incorrecto de algún determinado fenómeno o evento \cite{thompson}. 

Pueden surgir al existir una dificultad real en la comprensión del tema en sí o bien porque existe información conflictiva entre distintas fuentes, tales como pares, padres, docentes y otros medios a través de los cuales puede accederse a dicha información. 

Son razonamientos organizados, dotados de una lógica propia y que tienen una cierta coherencia que los hace verosímiles pero al mismo tiempo están alejados del funcionamiento correcto del fenómeno que intentan explicar \cite{johsua}. 

Algunos de los problemas que presentan es que son difíciles de identificar, una vez que se forman es complicado erradicarlas y son resistentes a la corrección, lo que dificulta la enseñanza y el aprendizaje de nuevos conocimientos. 

Conocer las \textit{misconceptions} a la hora de enseñar permite tener en cuenta las ideas con la que las personas parten y sirve de base para construir estrategias de enseñanza más efectivas.

Diversos autores y autoras han abordado el estudio de las \textit{misconceptions} en distintas disciplinas y áreas. En Física, una de las ideas alternativas más frecuentes en los alumnos, relativa a la caída de los cuerpos, es la suposición de que la velocidad que llevan al llegar al suelo depende del peso o la masa de estos \cite{ayensa}. Otro ejemplo de estudio de \textit{misconceptions} pertenece a la rama de la Biología. Giordan y De Vecchi trabajaron en esclarecer las ideas de niños y niñas de entre 12 y 14 años sobre la fecundación y el papel de las células sexuales en la misma, encontrando que el 75\% de los alumnos pensaban que el futuro bebé se encuentra contenido en el esperma o el espermatozoide \cite{johsua}.

En las Ciencias de la Computación la mayoría de estos estudios se han centrado en las \textit{misconceptions} sobre programación y en menor medida en otros temas del área, tales como Internet y sus servicios. Respecto a este tipo de estudios, no hemos encontrado en nuestra búsqueda publicaciones a nivel nacional que trataran estos tópicos.

En este contexto se enmarca este trabajo, en el que tratamos de investigar qué \textit{misconceptions} presentan niños y niñas de alrededor de 10 años de escuelas argentinas en algunos temas de Ciencias de la Computación, dejando de lado las \textit{misconceptions} sobre programación, que ya han sido más exploradas. En cambio, tratamos de enfocarnos en tópicos del área que se encuentran muy presentes en la cotidianeidad de los encuestados. 

De esta forma, las preguntas de investigación fueron:
\begin{enumerate}
\item ¿Existen \textit{misconceptions} relacionadas al almacenamiento de grandes volúmenes de datos en YouTube?
\item ¿Existen \textit{misconceptions} en torno a lo que sucede cuando se envía un archivo por WhatsApp? ¿Se genera una copia del mismo? ¿Quién tiene propiedad sobre el archivo enviado?
\item ¿Existen \textit{misconceptions} sobre la forma en la que se envían los mensajes por WhatsApp cuando no hay Wi-Fi?
\item ¿Existen \textit{misconceptions} relacionadas a la gratuidad de ciertas aplicaciones en Internet, como Instagram, TikTok o Facebook?
\end{enumerate}

Para responder estas preguntas, en primer lugar recopilamos las publicaciones de autores que han investigado sobre \textit{misconceptions} en temas de Ciencias de la Computación para entender qué tipo de estudios se han realizado y con qué resultados. A continuación, analizamos los consumos tecnológicos de chicos y chicas argentinos, escolarizados y de alrededor de 10 años, para decidir qué \textit{misconceptions} tratar. Luego, ideamos un cuestionario online que difundimos en algunas escuelas primarias del país para que fueran completados por los alumnos y alumnas de los cursos correspondientes durante la clase virtual. Con la información obtenida, realizamos un análisis cuantitativo y estadístico que nos permitió responder estas preguntas y formular a la vez nuevas preguntas de investigación que podrían continuar este trabajo.